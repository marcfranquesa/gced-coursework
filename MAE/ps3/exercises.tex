\exercise{Generation of random points}

\paragraph{Explanation:}
I mainly just followed the provided instructions and implemented the example use cases.
Instead of implementing more use cases I decided to be extra robust with my error handling.
This has led to quite a lengthy code, given all possible things that could go wrong.
I did my best to provide good error descriptions as well as correct usage information using function \mttext{get\_correct\_usage\_message} found at the end.
Additionally, the function has been separated in several handlers to simplify code comprehension.
This has led to me implementing the function using \mttext{varargin} to be able to pass around all arguments.

\paragraph{MATLAB Code:}

\begin{tiny}
    \verbatiminput{code/randpoints.m}
\end{tiny}

\paragraph{Results:}
Lets first start with an example error.
As can be seen an accurate description is provided as well as a full explanation of how it is used.

\begin{verbatim}
>> randpoints(-1)

Error using randpoints (line 33)

Value error, amount of random points (N)
has to be an integer strictly larger than 0.

  Syntax
	P = randpoints(N)
	P = randpoints(N, distr)
	P = randpoints(N, distr, U, V)

  Input Arguments
	N - amount of random points
	distr - distribution of points
	  'uniform' (default) | 'uniform'
	U - x-axis interval for distribution of points if distr is
	'uniform', mean of x, y if distr is 'gaussian'
	  numeric array of length 2
	V - x-axis interval for distribution of points if distr is
	'uniform', std of x, y if distr is 'gaussian'
	  numeric array of length 2
\end{verbatim}

Now lets show a couple of successful use cases.

\begin{verbatim}
>> randpoints(3)

ans =

    0.1332    0.3909    0.8034
    0.1734    0.8314    0.0605

>> randpoints(3, 'gaussian')

ans =

   -1.4286   -0.5607    1.1385
   -0.0209    2.1778   -2.4969

>> randpoints(5, 'uniform', [5 10], [-2 4])

ans =

    8.6893    6.3456    7.1142    7.7394    9.7137
    0.5065    3.8983   -0.1913    2.2066    1.9980
\end{verbatim}

Finally, assuming one would like to find out how the function works, we can use \mttext{help randpoints}.
I copied the structure used in \mttext{help rand} seeing as it was a similar function.

\begin{verbatim}
>> help randpoints
  randpoints - Random points generator
    This MATLAB function returns a 2xN matrix of random points
    according to the input parameters.
 
    Syntax
      P = randpoints(N)
      P = randpoints(N, distr)
      P = randpoints(N, distr, U, V)
 
    Input Arguments
      N - amount of random points
        positive integer value
      distr - distribution of points
        'uniform' (default) | 'uniform'
      U - x-axis interval for distribution of points if distr is
      'uniform', mean of x, y if distr is 'gaussian'
        numeric array of length 2
      V - y-axis interval for distribution of points if distr is
      'uniform', std of x, y if distr is 'gaussian'
        numeric array of length 2
\end{verbatim}


\paragraph{Comments:}
I was not expecting such a long implementation to accurately handle all cases.
It is possible it could be done it less lines but I tried my best.
A notable mention here is that unfortunately, due to the helper functions, some errors are called from inside of them which adds extra information to the errors, example below. 

\begin{verbatim}
>> randpoints(5, 'uniform', [5 10])
Error using randpoints>handle_bad_n_args (line 75)

Too many or not enough args please use 1, 2 or 4 args.

  Syntax
	P = randpoints(N)
	P = randpoints(N, distr)
	P = randpoints(N, distr, U, V)

  Input Arguments
	N - amount of random points
	distr - distribution of points
	  'uniform' (default) | 'uniform'
	U - x-axis interval for distribution of points if distr is
	'uniform', mean of x, y if distr is 'gaussian'
	  numeric array of length 2
	V - x-axis interval for distribution of points if distr is
	'uniform', std of x, y if distr is 'gaussian'
	  numeric array of length 2


Error in randpoints (line 26)
    handle_bad_n_args(varargin{:});
    ^^^^^^^^^^^^^^^^^^^^^^^^^^^^^^
\end{verbatim}

As can be seen at the bottom, information on what function caused it is displayed, which ideally I would have liked to not have.


\bigskip
\hrule


\exercise{Convex hull}

\paragraph{Explanation:}
I start by checking the input matrix is of the desired type.
Then initialize the algorithm as indicated in the statement and proceed with the algorithm also as indicated.
One thing I noticed however, is that with the particular degenerated case of \mttext{[1 1 1; 1 2 3]} running the function I got \mttext{[3     2     1     2     3]}.
Leading me to set all the previous points with minimum angle to $2 \pi $ and not just the current $P_0$.
Additionally I also added useful error messages as well as defined docstrings for the function.

\paragraph{MATLAB Code:}

\begin{tiny}
    \verbatiminput{code/convexhull.m}
\end{tiny}

\paragraph{Results:}

Lets start with the example in the statement.

\begin{verbatim}
>>  P = [1,3,7,5,5,6,8,3,8,4,8,0,5,4,1,9,8,5,5;
           7,1,3,5,0,6,2,3,0,3,1,6,2,7,0,2,5,3,1];

>> convexhull(P)
ans =
    16    17     6    14     1    12    15     5     9    16
\end{verbatim}

Lets now show an example error.

\begin{verbatim}
>> convexhull([1 2 3 1; 1 2 4 1])
Error using convexhull (line 29)

Value error, points matrix (P) has to be of size 2xN
and contain at least 3 unique pairs of real numbers.

  Syntax
	H = convexhull(P)
  Input Arguments
	P - random points
	distr - distribution of points
	  a matrix of size 2xN containing real numbers
      representing unique points on the plane, at
      least 3 points required
\end{verbatim}

\paragraph{Comments:}
I ran into a couple of self made bugs that slowed me down a bit.
Aside from that the main difficulty was initially understanding how the algorithm presented works.

\bigskip
\hrule

